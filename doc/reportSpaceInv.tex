\documentclass{article}
\usepackage[utf8]{inputenc}
\usepackage{amsmath}
\usepackage{graphicx}
\usepackage[top=1.55cm, bottom=2.29cm, left=1.6cm, right=1.47cm]{geometry}
\usepackage{fancyhdr}
\lhead{}
\chead{}
\rhead{Implementación de Space Invaders en FPGA}
\pagestyle{fancy}
\begin{document}

%%%% FRONTPAGE %%%%%%%%%%%%%%%%%%%%%%%%%%%%%%%%%%%%%%%%%%%%%%%%%%%%%%%%%%%
\begin{center}
\includegraphics[width=0.25\textwidth]{./uc3m.jpg}\\[2cm]
\textsc{\LARGE Universidad Carlos III de Madrid}\\[0.5cm]
\textsc{\Large Diseño de Circuitos Digitales}\\[4cm]


% Title
{\huge \bfseries{Implementación de Space Invaders en FPGA}\\[8cm]}


% Author and supervisor
\begin{minipage}{0.55\textwidth}
\begin{flushleft} \large
\emph{Authors:}\\
David Estévez Fernández\\
Sergio Vilches Expósito\\
\end{flushleft}
\end{minipage}
\begin{minipage}{0.4\textwidth}
\begin{flushright} \large
\emph{Teacher:}\\
Anna Vaskova
\end{flushright}\end{minipage}\vfill

% Bottom of the page
{\large \today}

\end{center}
%
\newpage
%
%%%%%%Table of contents%%%%%%%%%%%%%%%%%%%%%%%%%%%%%%
%%%%%%%%%%%%%%%%%%%%%%%%%%%%%%%%%%%%%%%%%%%%%%%%%%%%%
\tableofcontents
\newpage

\section{Introducción}
%
\section{Funcionamiento general}
%
\section{Explicación de los bloques}
%
\section{Conclusión / Posibles mejoras}
%
\end{document}